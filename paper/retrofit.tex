\section{Retrofitting languages with IFC}
\label{sec:retrofit}

We consider language of the form |(S, E, e, ->)|, where |S| is some
global state, |E| is an evaluation context, |e| is an expression in
the language and |->| is the reduction relation on the state,
evaluation context and expression;
we say expression |e| in context |E| and state |S| reduces to
expression |e'| in context |E'| and state |S'| and write
|conf S (E[e]) -> conf S' (E'[e'])|.
$|->| : (|S| \times |E| \times |e|) \rightharpoonup (|S| \times |E| \times |e|)$


To get the flavor for our approach, we will first show how our embedding
works for a simple, ML-like language, with references and general recursion.

\begin{figure}
\begin{code}
tv   ::= \tx.te | true | false | ta
te   ::= tv | tx | te te | if te then te else te | ref te | !te | te := te | diverge
tE   ::= tHole | tE te | tv tE | if tE then te else te
       | ref tE | !tE | tE := te | tv := tE 

te1 ; te2            @= (\tx.te2) te1  where  tx notmember FV(te2)
let tx = te1 in te2  @= (\tx.te2) te1
\end{code}

\begin{mathpar}
\and
\inferrule[T-pure]
{| te ~> te'|}
{| conf tS (iniE te) -> conf tS (iniE te') |}

\and
\inferrule[T-app]
{ } {| (\x.te) tv ~> {tv/x} te |}

\and
\inferrule[T-ifTrue]
{ } {| if true then te1 else te2  ~> te1|}

\and
\inferrule[T-ifFalse]
{ } {| if false then te1 else te2  ~> te2|}

\and
\inferrule[T-ref]
{| tS' = tS [ ta mapsto tv ] | } 
{| conf tS (intE (ref tv)) -> conf tS' (intE ta) |}
{| fresh (ta) |}

\and
\inferrule[T-deref]
{| tv = tS(ta) |}
{| conf tS (intE (!ta)) -> conf tS (intE tv) |}

\and
\inferrule[T-ass]
{| tS' = tS [ ta mapsto tv ] |} 
{| conf tS (intE (ta := tv)) -> conf tS' (intE tv) |}
\end{mathpar}

\caption{Simple ML-like language}
\label{fig:ml}
\end{figure}

\begin{figure}
\begin{code}
lop  ::=  canFlowTo | lub | glb
iv   ::=  id | il | true | false | unit
ie   ::=  iv | ix | ie lop ie | getLabel | setLabel ie
       |  fork ie | send ie ie ie | ix1, ix2 <= recv. ie ie |  IT te
iE   ::=  iHole | iE lop ie | iv lop iE | setLabel iE | fork iE 
       |  send iE ie ie | send iv iE ie | send iv iv iE
       |  ix1, ix2 <= recv. ie ie |  IT tE
io   ::=  (il, id ie)
Q    ::=  nil | io , Q

te   ::=  cdots | TI ie
tE   ::=  cdots | TI iE


ix1, ix2 <= brecv . ie @= IT (let tx = TI (ix1, ix2 <= recv. ie  (IT tx)) in tx)
\end{code}

\begin{mathpar}
\inferrule[I-step]
{|
conf tS (te) -> conf tS' te'
|}
{|
iconf iS (lconf id il tS (iniE (IT te)), ldots)
.->
iconf iS (lconf id il tS' (iniE (IT te')), ldots)
|}

\and
\inferrule[I-getLabel]
{ }
{|
iconf iS (lconf id il tS (iniE (getLabel)), ldots)
.->
iconf iS (lconf id il tS (iniE il), ldots)
|}

\and
\inferrule[I-setLabel]
{ |il canFlowTo il'| }
{|
iconf iS (lconf id il tS (iniE (setLabel il')), ldots)
.->
iconf iS (lconf id il' tS (iniE unit), ldots)
|}

\and
\inferrule[I-fork]
{ 
|iS' = iS [ id' mapsto nil ]|\\
|tS' = q(tS)|\\
|fresh (id')|
}
{|
iconf iS (lconf id il tS (iniE (fork ie)), ldots)
.->
iconf iS' (lconf id il tS (iniE id'), ldots, lconf id' il tS' (TI ie))
|}
%{|fresh (id')|}

\and
\inferrule[I-send]
{
|il canFlowTo il'|\\
|iS(id') = Q|\\
|iS' = iS [ id' mapsto (il', id,  ie) , Q ]|
}
{|
iconf iS (lconf id il tS (iniE (send id' il' ie))), ldots)
.->
iconf iS' (lconf id il tS (iniE unit), ldots)
|}

\and
\inferrule[I-recv]
{
| filter il (iS(id)) = io1 , ... , iok , (il', id', ie')|\\
| iS' = iS[id mapsto io1 , ... , iok]|\\
| ie1'' = {ie'/ix1,id'/ix2} ie1|
}
{|
iconf iS (lconf id il tS (iniE (ix1, ix2 <= recv. ie1 ie2)), ldots)
.->
iconf iS' (lconf id il tS (iniE ie1''), ldots)
|}

\and
\inferrule[I-noRecv]
{
| filter il (iS(id)) = nil|\\
| iS' = iS[id mapsto nil]|
}
{|
iconf iS (lconf id il tS (iniE (ix1, ix2 <= recv. ie1 ie2)), ldots)
.->
iconf iS' (lconf id il tS (iniE ie2), ldots)
|}

\inferrule[I-pure]
{| ie .~> ie'|}
{|
  iconf iS (lconf id il tS (iniE ie))
.->
iconf iS (lconf id il tS (iniE ie))
|}

\and
\inferrule[I-labelOp]
{ |denot (il lop il') = iv|}
{ |il lop il' .~> iv| }

\and
\inferrule[I-border]
{ } {| IT (TI (ie)) .~> ie |}

\and
\inferrule[T-border]
{ } {| intE (TI (IT (te))) ~> intE te |}
\end{mathpar}
\caption{IFC language nad general embedding.}
\label{fig:ml}
\end{figure}

The filter function |filter il Q| removes message whose labels do not
flow to |il|. Specifically, if |Q| is the empty list |nil|, the
function is simply the identity function, i.e.,
|filter il nil = nil|; otherwise:
\[
|filter il ((il', id, ie) , Q )| = \left\{
\begin{array}{l l}
|(il', id, ie) , filter il Q| & \quad \text{if |il' canFlowTo il|}\\
|filter il Q| & \quad \text{otherwise}
\end{array} \right.
\]


In the case of extending
\begin{mathpar}
\inferrule[I-bool]
{ |b member {true, false}|  } {| iniE (IT b) -> iniE b |}
\and
\inferrule[T-bool]
{ |b member {true, false}| } {| intE (TI b) -> intE b |}
\end{mathpar}




%% \cut{
%% (Selected rules from evaluation semantics. It's typed in the usual way.)
%% 
%% (Maybe this section needs to get moved) Sketch how a traditional account
%% of IFC would go about extending these rules.  These things are:
%% 
%% \begin{enumerate}
%%     \item Extend values with labels. (Values are isolated in runtimes,
%%         so runtime is labeled not values)
%%     \item Extend semantics to have a label on the program counter.
%%         (Ditto, runtime is values.)
%%     \item Extend semantics for assignment, as addresses need labels.
%%         (Labeled references are internalized in IFC language, and
%%         traditional references are isolated in runtimes.)
%%     \item Flow-sensitivity/insensitivity? (We choose flow-insensitive,
%%         flow sensitivity by reduction.)
%% \end{enumerate}
%% 
%% % No longer describing a "minimal non-Turing complete IFC calculus",
%% % instead, we jump straight to the combination.
%% 
%% We'd like to construct a coarse-grained IFC system on top of our language.
%% Intuitively, programs written in the target language are running on top of
%% an IFC ``operating system''.  The way we will perform our embedding is
%% by allowing programs to perform ``syscalls'' into the IFC runtime, and
%% vice-versa.  The IFC runtime is responsible for performing label checks
%% and ensuring TSNI is not violated.
%% 
%% We achieve this using a Matthews-Findler style embedding~\nocite{} to combine
%% the IFC language from the previous section with our example language.  Key points:
%% 
%% \Red{EZY reviewer hat: why is this type mediation necessary? It seems a bit
%% over-the-top.}
%% 
%% \begin{itemize}
%%     \item Expression and evaluation contexts are considered separately,
%%         so that some of our operational rules are easier to write
%%     \item IT/TI mediate cross-language type boundary (mention that this
%%         is where monitor failure/exceptions get handled)
%% \end{itemize}
%% 
%% First, we associate a label with every evaluation context.  Suppose you had
%% a function \verb|readFile|, you need to raise your label to write this
%% kind of program.
%% 
%% Motivate why we have labeled values. How to I avoid label creep problem? We
%% have fork. \Red{EZY shouldn't be in the ``background'' section.}
%% 
%% Background talks about IFC calculus in general; here we talk about
%% embedding and give the precise semantics for the IFC embedding.
%% 
%% \Red{Give a fake rule with how you raise your label when you read from a file.
%% We give you labeled values explicitly.}
%% 
%% \Red{EZY Why wouldn't a reviewer say, this sendMessage/receiveMessage thing
%% is pretty weird, why'd you do the API that way?}
%% 
%% \begin{verbatim}
%% % colored differently
%% e ::= fork e
%%     # let's do channels?
%%     | sendMessage e e  # asynchronous
%%     | receiveMessage e # registers the callback
%%     | label e e
%%     | unlabel e
%%     | IT e
%% 
%% v ::= tid
%%     | l
%%     | Labeled l v
%% \end{verbatim}
%% 
%% \Red{Prove type safety for combined language?}
%% 
%% \Red{Problem: synchronization primitives which are blocking. OK, talk
%% about this at the end. You can do whatever you want.}
%% % NB: the nice thing about sendMessage is we can just drop messages
%% % when the labels don't work out
%% %
%% % Intuitively, it would seem to me that any concurrency primitive
%% % would be fine, as long as you proved it safe, we don't really care.
%% 
%% (Full operational semantics. Provide type rules.)
%% 
%% \Red{What to do when you get stuck when you are about to invalidate IFC}
%% 
%% Which rules do we want to talk about?
%% 
%% \begin{enumerate}
%% \item fork: most complicated; shows side-condition (what do we do with stores on fork?)
%% \item label:  shows side-condition (what can we label? same as fork, but empty/all stores)
%% \item unlabel: shows raising of label
%% \item sendMessage: will give type for (maybe encapsulate queue in env)
%% \end{enumerate}
%% 
%% \subsection{The methodology (XXX fix title)}
%% \label{sec:methodology}
%% 
%% We now formally state our information-flow control transformation.  For
%% simplicity's sake, we consider only single-threaded source languages,
%% whose primary notion of execution is evaluation of an expression into a
%% value.  (Later, we consider how to accomodate systems which have
%% semantics for concurrent execution.)
%% 
%% \subsection{A tiny information-flow control calculus}
%% 
%% \begin{figure}[h]
%% \begin{align*}
%% \textrm{Thread \#:}   && \Ii          &   \\
%% \textrm{Label:}       && \Il          & \ \ \text{(we additionally use \Ic{} for clearances)}  \\
%% \textrm{Label op:}    && \ifc{\oplus} & ::=  \ifc{\lub}
%%                                         \ |\ \ifc{\glb}
%%                                         \ |\ \ifc{\flows}\\
%% \textrm{Value:}       && \Iv          & ::=  \Itrue 
%%                                         \ |\ \Ifalse
%%                                         \ |\ \Iunit
%%                                         \ |\ \Il
%%                                         \ |\ \Ierr
%%                                         \ |\ \Ii\\
%% %                                        \ |\ \IT{\Tv}{\Tstore}{\Tenv}\\
%% \textrm{Expression:}  && \Ie          & ::=  \Iv
%%                                         \ |\ \Ilop{\Ie}{\Ie}
%%                                         \ |\ \IgetLabel
%%                                         \ |\ \IgetClr\\&&&
%%                                         \ |\ \Ifork{\Ie}{\Ie}
%%                                         \ |\ \IT{\Te}\\
%% \textrm{Context:}     && \IE          & ::=  \Ihole{\ }
%%                                         \ |\ \Ilop{\IE}{\Ie}
%%                                         \ |\ \Ilop{\Iv}{\IE}\\&&&
%%                                         \ |\ \Ifork{\IE}{\Ie}
%%                                         \ |\ \Ifork{\Iv}{\IE}
%%                                         \ |\ \IT{\TE}\\
%% \textrm{Store (of target language):} && \Istore\\
%% \textrm{Prim label op:} && \denot{\cdot}   & \in \Ie \rightharpoonup \Iv \\
%% \end{align*}
%% % reduction rules
%% \begin{mathpar}
%% 
%% \inferrule[I-target]{
%%   \Tconf{\Tstore}{\Tenv}{\inTE{\Te}}
%%   \Tarrow
%%   \Tconf{\tar{\Tstore'}}{\Tenv'}{\inTE{\tar{\Te'}}}
%%   \\
%%   \TIstore{\Tstore}{\Tenv} = \Istore
%%   \\
%%   \TIstore{\tar{\Tstore'}}{\tar{\Tenv'}} = \ifc{\Istore'}
%% }
%% {
%%   \Iconf{\Istore}{\Il}{\Ic}{\inTE{\Te}}\Itasksep\cdots
%%   \Iarrow
%%   \Iconf{\ifc{\Istore'}}{\Il}{\Ic}{\inTE{\tar{\Te'}}}\Itasksep\cdots
%% }
%% 
%% \and
%% 
%% \inferrule[I-boundary]{ }
%% {
%%   \Iconf{\Istore}{\Il}{\Ic}{\inTE{\TI{\IT{\Te}}}}
%%   \Iarrow
%%   \Iconf{\Istore}{\Il}{\Ic}{\inTE{\Te}}
%% }
%% 
%% \and
%% 
%% \inferrule[I-label]{ }
%% {
%%   \Iconf{\Istore}{\Il}{\Ic}{\inTE{\TI{\IgetLabel}}}\Itasksep\cdots
%%   \Iarrow
%%   \Iconf{\Istore}{\Il}{\Ic}{\inTE{\TI{\Il}}}\Itasksep\cdots
%% }
%% 
%% \and
%% 
%% \inferrule[I-clearance]{ }
%% {
%%   \Iconf{\Istore}{\Il}{\Ic}{\inTE{\TI{\IgetClr}}}\Itasksep\cdots
%%   \Iarrow
%%   \Iconf{\Istore}{\Il}{\Ic}{\inTE{\TI{\Ic}}}\Itasksep\cdots
%% }
%% 
%% % fork
%% \inferrule[I-fork]
%% { 
%%   \denot{\Iflows{\Il}{\ifc{\Il'}}} = \Itrue\\
%%   \denot{\Iflows{\ifc{\Il'}}{\Ic}} = \Itrue\\
%%   \ifc{\Istore'} = \Tklone(\Istore)\\
%% }
%% {
%%   \Iconf{\Istore}{\Tenv}{\Il}{\Ic}{
%%     \inTE{\TI{
%%       (\Ifork{\Il'}{(\IT{\Te})})
%%     }}
%%   }
%%   \Itasksep\cdots
%%   %
%%   \Iarrow
%%   %
%%   \Iconf{\Istore}{\Il}{\Ic}{\inTE{\TI{\Ii}}}
%%   \Itasksep\cdots\Itasksep
%%   \Iconf{\ifc{\Istore'}}{\Il}{\Il}{\inTE{\Te}}
%%   \Itasksep\cdots
%% }{\operatorname{fresh}(\Ii)}
%% 
%% 
%% \end{mathpar}
%% 
%% \caption{IFC calculus}
%% \end{figure}
%% 
%% \begin{figure}[h]
%% \begin{align*}
%% \textrm{Address:}     && \Ta          & \\
%% \textrm{Variable:}    && \Tx          & \\
%% \textrm{Value:}       && \Tv          & ::= \Tclo{\Tlambda{\Tx}{\Te}}{\Tenv}\\
%% \textrm{Expression:}  && \Te          & ::=  \Tv
%%                                         \ |\ \Tlambda{\Tx}{\Te}
%%                                         \ |\ \Tx
%%                                         \ |\ \Te\ \Te
%%                                         \ |\ \Tset{\Te}{\Te}
%%                                         \ |\ \TI{\Ie}\\
%% \textrm{Context:}     && \TE          & ::=  \Thole{\ }
%%                                         \ |\ \TE\ \Te
%%                                         \ |\ \Tv\ \TE\\&&&
%%                                         \ |\ \Tset{\TE}{\Te}
%%                                         \ |\ \Tset{\Tx}{\TE}
%%                                         \ |\ \TI{\IE}\\
%% \textrm{Environment:} && \Tenv        & \in \Tx \rightharpoonup \Ta \\
%% \textrm{Store:}       && \Tstore      & \in \Ta \rightharpoonup \Tv \\
%% \textrm{Klone:}       && \Tklone   & \in \Tenv \times \Tstore \rightharpoonup \Tenv \times \Tstore \\
%% \end{align*}
%% % reduction rules
%% \begin{mathpar}
%% 
%% % lambda to closure
%% \inferrule[T-lam]{ }
%% {
%% \Tconf{\Tstore}{\Tenv}{\inTE{\Tlambda{\Tx}{\Te}}}
%% \Tarrow
%% \Tconf{\Tstore}{\Tenv}{\inE{\Tclo{\Tlambda{\Tx}{\Te}}{\Tenv}}}
%% }
%% 
%% \and
%% 
%% % lookup variable
%% \inferrule[T-var]{ }
%% {
%% \Tconf{\Tstore}{\Tenv}{\inTE{\Tx}}
%% \Tarrow
%% \Tconf{\Tstore}{\Tenv}{\inE{\Tstore(\Tenv(\Tx))}}
%% }
%% 
%% \and
%% 
%% % beta reduction
%% \inferrule[T-$\beta$]
%% {
%% \tar{\Tstore'}=\Tstore[\Ta \mapsto \Tv]\\
%% \tar{\Tenv''} =\tar{\Tenv'}[\Tx \mapsto \Ta]
%% }
%% {
%% \Tconf{\Tstore}{\Tenv}{\inTE{\Tclo{\Tlambda{\Tx}{\Te}}{\tar{\Tenv'}}\ \Tv}}
%% \Tarrow
%% \Tconf{\tar{\Tstore'}}{\tar{\Tenv''}}{\inE{\Te}}
%% }{\operatorname{fresh}(\Ta)}
%% 
%% \and
%% 
%% % set!
%% \inferrule[T-set]
%% {
%% \tar{\store'} = \Tstore[\Tenv(\Tx) \mapsto \Tv]
%% }
%% {
%% \Tconf{\Tstore}{\Tenv}{\inTE{\Tset{\Tx}{\Tv}}}
%% \Tarrow
%% \Tconf{\tar{\store'}}{\Tenv}{\inE{\Tv}}
%% }
%% % fork
%% %\inferrule[E-fork]
%% %{ 
%% %\Iflows{\Il}{\Il} \Ilabelarrow \Itrue\\
%% %\Iflows{\Il}{\Ic} \Ilabelarrow \Itrue\\
%% %\tar{\Tstore'} = \Tklone{\Tstore}(\Tstore)\\
%% %\tar{\Tenv'}   = \Tklone{\Tenv}(\Tenv)
%% %}
%% %{
%% %\Iconf{\Il}{\Ic}{\inIE{ \Ifork{\Il}{
%% %      (\downcall{(\upcall{^{\Tstore}_{\Tenv}\ \Ie})})
%% %}}}\|\cdots
%% %\to
%% %\Iconf{\Il}{\Ic}{\inE{\Ii}}\|\cdots\|
%% %\Iconf{\Il}{\Il}{\inE{
%% %}}\|\cdots
%% %}{fresh(\Ii)}
%% 
%% 
%% \end{mathpar}
%% \caption{Simple calculus}
%% \end{figure}
%% 
%% Our first task is to define a information-flow control mini-language
%% which will be combined with the source language. (Reference)  This
%% language is not Turing-complete; this is intentional, as (1) the ability
%% to call across the language boundary means that any control flow can be
%% implemented in the source language, and (2) it simplifies the
%% non-interference proof.  The language has a very simple set of typing
%% rules.  The operational semantics for this language are worth remarking
%% on (write a bit about how this IFC system works).  We can now state and
%% prove a simple non-interference theorem about this language:
%% 
%% \begin{theorem}
%% (Statement of noninterference)
%% \end{theorem}
%% \begin{proof}
%% XXX TODO should be simple
%% \end{proof}
%% 
%% \subsection{Combination (XXX)}
%% 
%% 
%% }
