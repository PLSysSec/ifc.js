%\cut{
\section{Conclusion}
\label{sec:conclusion}

We presented a formal semantics which describes how to extend arbitrary languages
with language-based coarse-grained IFC. We prove the security guarantees
TSNI and TINI, where the proof technique is parametrized on the
target language, for two scheduling policies. In
addition, we provided conditions on how to securely refine our formal semantics to
consider optimizations required in practice. Finally, we described extensions to
the core IFC language with more advanced concepts like labeled values, privileges,
and a notation of clearance, and therefore connecting it with existing
IFC implementations for Javascript, Haskell, and (in
less detail) proposing IFC for C. With this work, we aim to expose how ideas
from IFC in OS can be systematically applied in programming languages.


% In this paper, we develop a general coarse-grained approach to IFC
% based on dividing a system into relatively course computational units
% and tracking only communication between these isolated units.
% We give formal semantics for the core coarse-grained
% information flow control language and show how a large class of target languages
% can be combined with it to achieve provable
% non-interference.
% We then give a proof technique for showing  non-interference
% of a concrete semantics for a potentially optimized IFC language
% by means of an isomorphism, and identify a class of IFC language restrictions
% that preserves non-interference.
% We briefly describe ways to enrich the core IFC language with
% more advanced concepts such as labeled values, privileges, or a
% notation of clearance and
% connect our formal semantics to real implementations of
% coarse-grained IFC systems for Javascript, Haskell, and
% (in less detail) enforcing IFC in C.
% Leaving more detailed presentation to later work,
% we informally relate our results to a monadic interpretation
% of IFC which guided our design.
% }
