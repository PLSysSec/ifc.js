\begin{abstract}
Web sites could compromise users' privacy by accidentally incorporating
error-prone or malicious third-party JavaScript. Informa\-tion-flow control
(IFC) is a promising technique to secure web sites against untrusted JavaScript
code. Unfortunately, existing IFC approaches demand non-trivial modifications to
the JavaScript engine or a full instrumentation of the JavaScript semantics. In
contrast, and inspired by operating system (OS) research, we present an
alternative technique based on attaching a single IFC policy to a single
execution context, and therefore abstracting away from most of the internals of
JavaScript. Specifically, we provide a coarse-grained IFC monitor which does not
require any modifications to either JavaScript's semantics or its engine---the
approach only exposes a small API to JavaScript code in order to interact with
the monitor. To reason about our design principles, we describe a formal
framework which treat target languages, e.g. JavaScript, as black-boxes with a
clear interface to the monitor. We instantiate our framework to provide security
guarantees (i.e. non-interference) to an IFC monitor implemented in Firefox.
\end{abstract}
