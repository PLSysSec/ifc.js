\begin{abstract}
Web sites could compromise users' privacy by accidentally incorporating
error-prone or malicious third-party JavaScript. Informa\-tion-flow control
(IFC) is a promising technique to securely incorporate untrusted JavaScript into
web sites and it allows the safe integration of mutually distrustful websites
(mash-ups). Unfortunately, existing IFC approaches demand non-trivial
modifications to the JavaScript engine or a full instrumentation of the
JavaScript semantics in order to track how information flows inside web
applications. Inspired by operating system research, we present an alternative
technique based on attaching a single IFC policy to a single browsing context,
and therefore abstracting away from the internals of JavaScript. Specifically,
we provide a coarse-grained IFC monitor which does not require any modifications
to either JavaScript semantics or its engine---the approach only exposes a small
API to the JavaScript code in order to interact with the monitor. To reason
about our design principles, we describe a formal framework which treat target
languages (e.g. JavaScript) as black-boxes with a clear interface to the
monitor. We instantiate our framework to provide security guarantees
(i.e. non-interference) to an IFC monitor implemented in Firefox.
\end{abstract}
