\section{Security guarantees}
\label{sec:formal}

In this section, we present the formal security guarantees provided by
our embedding.
%
Concretely, we show that |specLang alpha targetLang|, with an appropriate
scheduler |alpha|, satisfies non-interference~\cite{Goguen82},
without making any reference to properties of |targetLang|.
%
The precise scheduling policy dictates
what guarantee we can achieve for programs with diverging tasks.
For a sequential scheduler |seqf|, we will only be able to show
\emph{termination-insensitive non-interference}, where a program
may diverge based on secret data; this allows attackers to observe
secrets by observing the termination of tasks.
For the concurrent round-robin schedule |roundrobinf|,
we can show a stronger result known as
\emph{termination-sensitive non-interference},
where termination attacks cannot leak information.

How can we characterize what the \emph{secret inputs} of a program are?  Like
other work~\cite{Li+:2010:arrows,Russo+:Haskell08,lio,stefan:addressing-covert},
we specify and prove non-interference using \emph{term erasure}.
%
Intuitively, term erasure allows us to show that an attacker does not learn
any sensitive information from a program if the program behaves identically
(from the attacker point of view) to a program with all sensitive data
``erased''.
%
To accommodate for this, we extend our formalism with a function |erasef l| that
performs erasures by mapping configurations to erased configurations,
usually by rewriting (parts of) configurations that are more sensitive
than |l| to a new syntactic construct |bullet|.
%
In order for us to state non-interference, a language needs to include an
erasure function:

%\Red{should be the environment colored in the definition?}
%EZY: No, I don't think so.
\begin{definition}[Information flow control language]
    An information flow control language |L| is a tuple |(C, .->,
    erasef l)|, where $|C|$ is the type of machine configurations (members
    of which are usually denoted by the meta-variable |c|), |.->| is a
    reduction relation between machine configurations and |erasef l : C -> tyerase C|
    is an erasure function parametrized on labels from machine configurations to \emph{erased} machine
    configurations |tyerase C|.  Sometimes, we refer as |V| to set of
     terminal configurations in |C|, i.e., configurations states where
     no further transitions are possible.
\end{definition}

%{
%format .->  = "\overset{\alpha}{\hookrightarrow}"
Our language |specLang alpha (targetLang nop)| can be
represented as |(ifc C, .->, erasef il)|, such that
$|ifc C| = |iS| \times \prod |it|$.  Here,
$|ifc V| = |iS| \times |it|_V$, where $|it|_V \subset |it|$ is the
type for tasks whose expressions have been reduced to
values.\footnote{
  Here, we abuse notation by describing types for configuration parts using the
  same metavariables as the ``instance'' of the type, e.g., |it| for the type of
  task.
}
The erased configuration |tyerase (ifc C)| extends |ifc C| with configurations
containing |bullet|, and Figure~\ref{fig:erasure} gives the precise definition for
our erasure function |erasef il|.
%
A task and its corresponding message queue is completely removed from the task
list if its label does not flow to the attacker observation level |il|.
Otherwise, we apply the erasure function homomorphically and remove any message
from the task's message queue that are more sensitive than |il|.
%}

\begin{figure} % fig:erasure
\begin{align*}
  &|erase il (iconf iS its) =
  iconf (erase il iS) (fltr (\ it . it = bullet) (map (erasef il) its))|\\
  &|fullconf id il' tS ie| \begin{cases}
    |bullet| & |il' cantFlowTo il| \\
    |fullconf id il' tS ie| & \text{otherwise}
  \end{cases} \\
  &|erase il emptyset = emptyset|\\
  &|erase il (bracket iS (mapsto id Q)) =| \begin{cases}
    |erase il iS| & \text{|il' cantFlowTo il|, where |il'| is}\\
    & \text{the label of thread |id|} \\
    |bracket (erase il iS) (mapsto id (erase il Q))| & \text{otherwise}
  \end{cases} \\
  &|erase il Q = filter il Q|\\
  &|erase il v = v|
\end{align*}
\caption{ Erasure function for tasks, queue maps, message queues, and
configurations.  In all other cases, including target-language constructs,
|erasef il| is applied homomorphically.  \label{fig:erasure} }
\end{figure}

The definition of |erasef il| is quite important, as it captures the IFC
language's attacker model: elements that are erased cannot be observed
by the attacker.
%
In our case, we assume that the attacker cannot observe sensitive tasks or
messages, or even the number of such entities.
%
While such assumptions are standard~\tocite{}, our definitions allow for
stronger attackers that may be able to inspect resource usage.\footnote{
  We believe that we can extend |specLang alpha (targetLang nop)| to
  such models using the resource limits techniques of~\cite{yangresource}.
  %
  We leave this extension to future work.
}

Non-interference relies on the definition of an attacker's observational power at
security level |l|, which is typically defined as an equivalence
relation---called |l|-equivalence---on configurations.  An attacker
at level |l| cannot
distinguish two configurations that are |l|-equivalent.
%
Since our erasure function captures the attacker model, we simply define this
equivalence as the syntactic-equivalence of erased configurations~\cite{stefan:addressing-covert}.
%
\begin{definition}[|l|-equivalence]
    In a language |(C, .->, erasef l)|, two machine configurations
    |memberf (c, c') C| are considered $l$-equivalent, written as |c ~= c'|,
    if |erase l c = erase il c'|.
\end{definition}
%

Intuitively, we can now state that a language satisfies non-interference if an
attacker at level |l| cannot distinguish the runs of any two |l|-equivalent
configurations.
%
This precise non-interference property is called termination sensitive non-interference
(TSNI).  Besides the obvious requirement to not leak secret information
to public channels, this definition also requires the termination
of public tasks to be independent of secret tasks.
%
Formally, we define TSNI as follows.

\begin{definition}[Termination Sensitive Non-Interference (TSNI)]
  A language |(C, .->, erasef l)| is termination
  sensitive non-interfering if for any label |l|, and configurations
  |c1, c1', c2 member C|, if
  \begin{equation} \label{eq:tsni-lhs}
    |c1| \approx_{|l|} |c2|
    \qquad \text{and} \qquad
    |c1| |.->|^* |c1'|
  \end{equation}
  then there exists a configuration |c2' member C| such that
  \begin{equation} \label{eq:tsni-rhs}
    |c1'| \approx_{|l|} |c2'|
     \qquad \text{and} \qquad
    |c2| |.->|^* |c2'|
    \ \text{.}
  \end{equation}
\end{definition}
%
In other words if we take two |l|-equivalent configurations, then for every
intermediate step taken by the first configuration, there is a corresponding
number of steps that the second configuration can take to result in a
configuration that is |l|-equivalent to the first resultant configuration.


We now show that our IFC language satisfies TSNI %under a particular
%scheduling policy.
%
%Specifically, we consider the language 
under the round-robin scheduler
|roundrobinf| given in Figure~\ref{fig:scheduler}.

\begin{theorem}[Concurrent IFC language is TSNI]
  \label{thm:rr-tsni}
For any target language |targetLang|, |specLang roundrobinf
targetLang| satisfies TSNI.
\end{theorem}

In general TSNI will not hold for an arbitrary scheduler |alpha|.
%
For example, |specLang alpha targetLang| with a scheduler that inspects a
sensitive threads current state when deciding which thread to schedule next
will in general break non-interference~\cite{Russo:Sabelfeld:CSFW06,BartheRRS07}.
%
%In Section~\toref{} we detail the preceise scheduler for which our
%specification language is TSNI.
%

However, even non-adversarial schedulers are not always safe.
Consider, for example, the sequential scheduling policy |seqf| given in
Figure~\ref{fig:scheduler}.
%
It is easy to show that |specLang seqf targetLang| does not satisfy
TSNI.
%
Concretely, consider a target language similar to |targetLangML| with an
additional expression terminal |diverge| that denotes a divergent computation,
i.e., |diverge| always reduces to |diverge| and a simple label lattice |{pub,
sec}| such that |pub canFlowTo sec|, but |sec cantFlowTo pub|.
Consider the following two configurations in this language:
\begin{code}
ic1 = iS ; fullconf 1 sec tS1 (IT(if false then diverge else true)),  fullconf 2 pub tS1 ie
ic2 = iS ; fullconf 1 sec tS1 (IT(if true then diverge else true)),   fullconf 2 pub tS2 ie
\end{code}
\Red{Ale: |when| was not defined, so I changed it to if-then-else}
These two configurations are |pub|-equivalent, but |ic1| will reduce 
(in two steps) 
to |ic1 = iS ; fullconf 2 pub tS1 (IT true)|, whereas |ic2| will not make
any progress.
%
Suppose that |ie| is a computation that writes to a |pub| channel,\footnote{
Though we do not model labeled channels, extending the calculus with such a
feature is straight forward, see Setion~\ref{sec:extensions}.}
then the |sec| tasks's decision to diverge or not is directly leaked to a
public entity.

% sheule: I feel like this forward reference is not useful here.
%In certain cases, however, a TSNI-safe concurrent scheduling policy such as
%|roundrobinf| cannot be considered without imposing a non-trivial burden on the
%concrete implementation.
%(Section~\ref{ref:real} describes a particular embedding with JavsScript as the
%target language.)
%
To accomodate for sequential languages, or cases where a weaker guarantee
is sufficient, we consider an alternative non-interference property called termination insensitive
non-interference (TINI).  This property can also be upheld by sequential languages at the cost
of leaking through (non)-termination~\cite{Askarov:2008}.
%
\begin{definition}[Termination insensitive non-interference (TINI)]
  A language |(C, V, .->, erasef l)| is termination
  insensitive non-interfering if for any label |l|, and configurations
  |c1, c2 member C| and |c1', c2' member V|, it holds that
  \[
    (|c1| \approx_{|l|} |c2|
    \;\;\land\;\;
    |c1| |.->|^* |c1'|
    \;\;\land\;\;
    |c2| |.->|^* |c2'|)
    \implies
    |c1'| \approx_{|l|} |c2'|
  \]
\end{definition}

TSNI states that if we take two |l|-equivalent configurations, and both
configurations reduce to final configurations (i.e.,
configurations for which there are no
possible further transitions), then the end configurations are also
|l|-equivalent.
%
We highlight that this statement is much weaker than TSNI: it only states that
terminating programs do not leak sensitive data, but makes no statement
about non-terminating programs.

As shown by compilers~\cite{jif,FlowCaml}, interpreters~\cite{JSFlow}, and
libraries~\cite{Russo+:Haskell08,lio}, TINI is useful for sequential
settings. We show that our IFC language with the sequential scheduling policy
|seqf| satisfies TINI.
%
\begin{theorem}[Sequential IFC language is TINI]
  \label{thm:seq-tini}
For any target language |targetLang|, |specLang seqf targetLang| satisfies TINI.
\end{theorem}



% Local Variables:
% TeX-master: "main.tex"
% TeX-command-default: "Make"
% End:
