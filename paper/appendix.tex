\appendix


\section{Formal Extensions}

\begin{figure}
  %{
  %format .->  = "\overset{\alpha}{\hookrightarrow}"
  %format .-/>  = "\not\overset{\alpha}{\hookrightarrow}"
  \begin{code}
    iv  ::= cdots | Labeled il ie
    ie  ::= cdots | label ie ie | unlabel ie | labelOf ie
    iE  ::= cdots | label iE ie | unlabel iE | labelOf iE
  \end{code}
  \begin{mathpar}
    \inferrule[I-label]
    {|
      il canFlowTo il'
    |}
    {|
      niconf iS id il (label il' ie)
      ->
      niconf iS id il (Labeled il' ie)
    |}
    \and
    \inferrule[I-unlabel]
    {|
      il''=il lub il'
    |}
    {|
      niconf iS id il (unlabel (Labeled il' ie))
      ->
      niconf iS id (il'') ie
    |}
    \and
    \inferrule[I-labelOf]
    {|
    |}
    {|
      niconf iS id il (labelOf (Labeled il' ie))
      ->
      niconf iS id il (il')
    |}
  \end{mathpar}
  \caption{}
  \label{fig:labeled-vals}
  %}
\end{figure}

\begin{figure}
  %{
  %format .->  = "\overset{\alpha}{\hookrightarrow}"
  %format .-/>  = "\not\overset{\alpha}{\hookrightarrow}"
  \begin{code}
    iv  ::= cdots | ial
    ie  ::= cdots | new ie ie | read ie | write ie ie
    iE  ::= cdots | new iE ie | new il iE | read E
                  | write iE ie | write ial iE
    iS  ::=  cdots | bracket iS (mapsto ial iv)
  \end{code}
  
  \begin{mathpar}
    \inferrule[I-new]
    {
      |il canFlowTo il'| \\
      |fresh ia|\\
      |iS'= bracket iS (mapsto (iall il') iv)|
    }
    {|
      niconf iS id il (new il' iv)
      ->
      niconf iS' id il (iall il')
      |}
    \and
    \inferrule[I-read]
    {}
    {|
      niconf iS id il (read ial')
      ->
      niconf iS id (il lub il') (readState iS (iall il'))
      |}
    \and
    \inferrule[I-write]
    {
      |il canFlowTo il'|\\
      |iS'= bracket iS (mapsto (iall il') iv)|
    }
    {|
      niconf iS id il (write ial' iv)
      ->
      niconf iS' id il (unit)
      |}
    \and
    \inferrule[I-labelOf2]
    {}
    {|
      niconf iS id il (labelOf ial')
      ->
      niconf iS id il (il')
      |}
  \end{mathpar}
  \caption{}
  \label{fig:labeled-refs}
  %}
\end{figure}


\begin{figure}
  \begin{align*}
  &|erase il (iconf iS its) =
  iconf (erase il iS) (fltr (\ it . it = bullet) (map (erasef il) its))|\\
  &|fullconf id il' tS ie| \begin{cases}
  |bullet| & |il' cantFlowTo il| \\
  |fullconf id il' (erase il tS) (erase il ie)| & \text{otherwise}
  \end{cases} \\
  &|erase il (Labeled il' ie)|= \begin{cases}
  |Labeled il' bullet| & |il' cantFlowTo il| \\
  |Labeled il' ie| & \text{otherwise}
  \end{cases} \\
  &|erase il emptyset = emptyset|\\
  &|erase il (bracket iS (mapsto id Q)) =| \begin{cases}
  |erase il iS| & \text{|il' cantFlowTo il|, where |il'| is}\\
  & \text{the label of thread |id|} \\
  |bracket (erase il iS) (mapsto id (erase il Q))| & \text{otherwise}
  \end{cases} \\
  &|erase il (bracket iS (mapsto (iall il') iv)) =| \begin{cases}
  |bracket (erase il iS) (mapsto (iall il') bullet)| & \text{|il' cantFlowTo il|}\\
  |bracket (erase il iS) (mapsto (iall il') (erase il iv))| & \text{otherwise}
  \end{cases} \\
  &|erase il Q = filter il Q|
  \end{align*}
  \caption{Erasure function for the full IFC language, with all extensions.
    In all cases not specified (including target-language constructs),
    |erasef il| is applied homomorphically.
    This definition replaces the one from Figure~\ref{fig:erasure}, which
    is for the IFC language without extensions.}
  \label{fig:erasure2}
\end{figure}


\section{Full Non-Interference Proof}
\label{sec:appendix}

In the main body of the paper we have only sketched the proof structure
for Lemma~\ref{lemma:rr-tsni-general}.  Here we give it with full
details.

\begin{proof}[Proof of Lemma~\ref{lemma:rr-tsni-general}]
  First, we observe there must be at least one task in |ic1|, otherwise
  it could not take a step.  Thus, |ic1| is of the form
  |iconf iS1 (it1, its1)|.  Consider two cases:
  \begin{itemize}
    \item $|erase il it1|=|bullet|$.
    By the definition of |erasef il|, we know that |il canFlowTo lcurr|
    where |lcurr| is the label of |it1|.
    In this case, we do not need to take a step for
    |ic2|, because |ic2'=ic2| will already be |il|-equivalent to |ic1'|.
    To see that, note that the tasks |its1| in |ic1| are left in the
    same order and unmodified (the scheduling policy only
    modifies the first task). The task |it1| either
    gets dropped (by \textsc{I-noStep}), or
    transforms into a task |it1'| as well as potentially spawning a new
    task |it1''|.  Since both |it1'| and |it1''| have a label that is
    at least as high as the label of |it1| (can be seen
    by inspecting all reduction rules), they will get filtered
    by |erasef il| in |ic1'|.  Therefore, the |il|-equivalence of the
    task list is guaranteed.
    Lets consider the possible changes to |iS1|:
    Only three reduction interact with |iS1|,
    thus it suffices to consider these cases:
    \begin{description}
      \item[Case \textsc{I-send}]
      A new message triple with label |il'| gets added to the message
      queue of the receiving thread.  However, since |lcurr canFlowTo il'|,
      the triple will get erased.
      \item[Case \textsc{I-recv} and \textsc{I-noRecv}]
      In this case, only the queue of
      task |it1| can change, which gets erased.
    \end{description}
    This ensures that $|ic1'|\approx_{|il|}|ic2'|=|ic2|$.
    %    \alphacondition{We need all scheduling policies to not change the order
    %      of any tasks (except for the first one).  Newly spawned task can appear
    %      anywhere in the list.}
    \item $|erase il it1|\neq|bullet|$.
    Here, there must be a corresponding
    task |it2| in |ic2|,
    such that |it1=it2| (otherwise |ic1| and
    |ic2| could not be |il|-equivalent).
    However, |it2| might not be at the beginning of the task list yet, but
    all tasks occurring before it must get erased by |erasef il|.
    In |ic2|, we can first take some number of steps until |it2| moves
    to the front of the list.
    This is the case regardless of any additional side conditions $P$ on
    rules, because for all of these tasks, it can either take an actual
    step, or it gets dropped by \textsc{I-noStep} (which is always
    possible, as there cannot be an additional side condition on this
    rule).  All tasks that didn't get dropped are still at a label
    that isn't below |il| and thus get erased.
    
    %    \alphacondition{The scheduling policy must eventually let any task in
    %      the task list evaluate.  In particular, it cannot get stuck when the
    %      first task gets stuck, or keep executing a small number of tasks
    %      exclusively forever (e.g. just execute the first task all the time
    %      if it gets into an infinite loop).}
    
    Therefore, after |ic2| has potentially executed some number of steps
    to arrive at |ic2''|, we are now in the situation where $|ic1|\approx_{|il|}|ic2''|$, and the first tasks |it1| and |it2|,
    respectively, don't get erased and are thus equivalent.
    The task |it2| can now take exactly the same step as |it1|;  this
    is true even with arbitrary additional premises $P$ that follow
    the condition in Definition~\ref{def:restricted}, since those
    predicates only depend on |erase il ic1|, which is equivalent
    to |erase il ic2|, and thus those predicates evaluate in the same way.
    Thus, we only
    need to argue that the potential differences in |iS1| and |iS2| cannot
    have an influence on the execution (and we know |iS1| and |iS2| are
    |il|-equivalent).
    Again, only sending and receiving will depend on, or change |iS|,
    so we consider all these cases.
    \begin{description}
      \item[Case \textsc{I-send}]
      Here, the task |it2| will send the same message to the same
      receiver queue. This
      queue is either completely erased, or it is |il|-equivalent.  In both
      cases, |il|-equivalence of |iS1'| and |iS2'| is preserved.
      \item[Case \textsc{I-recv} and \textsc{I-noRecv}]
      When the tasks are receiving a message, then by the reduction rules
      we know that they first filter the queue by the label
      |lcurr| of |it1|.  We
      also know that the queues are equivalent when filtered by the less
      restrictive label |il|, thus the messages received (or dropped) from the
      queue are equivalent.
    \end{description}
  \end{itemize}
\end{proof}
