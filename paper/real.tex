\section{Real World Languages}
\label{sec:real}

In the introduction, we stated that the original motivation behind
developing this embedding was in order to formally specify a
coarse-grained information flow control language for JavaScript.
%
In Section~\ref{sec:retrofit}, however, we used only a simplified
language as an example target language.
%
In this section, we consider the application of our embedding to
JavaScript and two other real world target languages, C and Haskell.
%
Our goal is to show the flexibility of the embedding in settings with
vastly different features and properties, as well as discuss the
semantic gap that must be overcome in some cases to apply our formalism
to other systems.
%
Two of the systems we describe have been implemented: the Haskell
system~\cite{lio} and the SWAPI~\cite{swapi} system; we leave the
implementation of the IFC system for C to future work.

\subsection{C}
\label{sec:real:c}
%
C programs are able to execute arbitrary (machine) code, access
arbitrary memory, and perform arbitrary syscalls.
%
Thus, the confinement of C programs must be imposed by the underlying OS
and hardware.
%
For instance, this isolation can be achieved using Dune's hardware protection
mechanisms~\cite{Belay:2012:DSU:2387880.2387913}, similar to a simpler
version of Wedge~\cite{Belay:2012:DSU:2387880.2387913,
Bittau:2008:WSA:1387589.1387611} with an information flow control
policy.
%
Using page tables, a (trusted) IFC runtime could ensure that each task,
implemented as a lightweight process, can only access the memory it
allocates--tasks do not have access to any shared memory.
%
In addition, ring protection could be used to intercept syscalls performed by
a task and only permit those corresponding to our IFC language (such as
|getLabel| or |send|).
%
Dunes hardware protection mechanism allows us to provide a concrete
implementation that is efficient and relatively simple to reason
about, but other sandboxing mechanisms could be used in place of Dune.

In this setting, the combined language of Section~\ref{sec:retrofit}
can be interpreted in the following way: calling from the target
language to the IFC language corresponds to invoking a syscall.
%
Creating a new task with the |sandbox| syscall is corresponds to
\emph{forking} a process.  Here, page tables limit what memory
a task can access: we can ensure there will be no memory (effectively
defining |klone(tS)
= tS0|, where |tS0| is the set of pages necessary to bootstrap a
lightweight process.)
%
Similarly, control over page tables and protection bits allows us to
define a |send| syscall that copies pages to our
(trusted) runtime queue; and, correspondingly, a |recv| that copies
the pages from the runtime queue to the (untrusted) receiver.
%
Since C is not memory safe, conditions on these syscalls are
meaningless.


\subsection{Haskell}
\label{sec:real:hs}
In contrast with the C embedding which relies on hardware protection
mechanisms, a Haskell implementation can leverage Haskell's strong data
abstraction and static type system, monadic approach to effects, and
lightweight concurrency to implement the embedding in a more lightweight
manner.  We briefly describe one such system, LIO~\cite{lio}, which
implements IFC as a library.

LIO is implemented by defining a new monad, \verb|LIO|, which wraps Haskell's \verb|IO|
monad.
%
The purpose of this monad is twofold: it restricts the use of
arbitrary effects that would ordinarily be allowed by the \verb|IO| monad,
and it associates labels with tasks.
%
Computations in the \verb|LIO|
monad can be thought to be operating within the IFC system.
%
One important aspect of using \verb|IO| as the base for this
implementation is that it allows use of Haskell's efficient
implementation of threads, channels, etc. (e.g., in the concurrent
version of LIO, we use Haskell's \texttt{forkIO} to fork a lightweight
thread in the case of |fork|~\cite{stefan:addressing-covert}), in
contrast to defining them in a completely pure fashion (as suggested
in Section~\ref{sec:monad}).

What is the interpretation of this system as per Section~\ref{sec:retrofit}?
%
Here, the \emph{pure subset} of Haskell is the target language, while
the monadic subset of Haskell in the \verb|LIO| monad is the IFC
language.
%
Crucially, the lack of unrestricted mutation in the pure fragment of
Haskell prevents direct communication between tasks, even when memory is
shared between them.\footnote{However, lazy evaluation can still produce
a covert channel.}

Since this IFC system is implemented as a library in Haskell,
implementation-wise, we must ensure that these languages are indeed the subsets of Haskell
we claim, i.e., while the concrete language is all of Haskell, we must
ensure that we can restrict programs to our subset of Haskell that
encodes the combined language.
%
To this end, \verb|LIO| relies on Haskell's strong data abstraction and type system
(as enforced by Safe Haskell~\cite{Terei:2012:SH:2364506.2364524}) to
ensure that arbitrary \verb|IO| actions cannot be lifted into
\verb|LIO|.
%
In other words, assuming \verb|LIO| is implemented correctly, programs
written in the \verb|LIO| monad cannot perform arbitrary \verb|IO| actions
without breaking abstraction.

We refer the interested reader to~\cite{lio,stefan:addressing-covert} for
additional details on the various implementations of this system.\footnote{It's worth noting that the proofs of non-interference we have given for asynchronous communication primitives are new and not in the original presentation of LIO.}


\subsection{JavaScript}
\label{sec:real:js}

JavaScript (JS), as specified by ECMAScript~\cite{ecma}, does not have any
built-in functionality for I/O. Capabilities such as the ability to
mutate the DOM and read user input are APIs defined on top of the ECMAScript
specification.
%
Consequently, and in contrast to our C and Haskell embeddings---for
which we must eliminate external effects---the embedding for JS is
trivial.

We describe SWAPI~\cite{swapi}, an implementation of this system for JavaScript
(which we here denote by |targetLangJS|).
%
The direct approach for implementing the embedded language |specLangJS roundrobinf| is
by running multiple instances of the JS runtime, i.e., in
separate OS-level threads.\footnote{
 The Firefox implementation of SWAPI also considers the case where
 tasks share the JavaScript runtime (and are cooperatively scheduled
 on the event-loop), but the heap and execution context of each task
 is disjoint. This is similar to the single-heap language of
 Section~\ref{sec:concrete}.  However, since the heap separation is
 provided by the browser the browser, we do not discuss this further.
 The interested reader is referred to~\cite{swapi} for more details.
}
%
In SWAPI, the IFC language functionality is implemented in the JS
runtime much like browser layout engines implement the DOM, and expose
the new functions, e.g., \verb|getElementById|, by attaching them to
the global JS object.
%
In turn, each task created with |sandbox| is
executed in a separate thread, running a separate, \emph{fresh} instance of our
modified JS runtime.
%
Formally, |sandbox| is defined with |klone(tS) =
tS0|, where |tS0| is the global object corresponding to the standard
JS library (e.g., |tS0| contains \texttt{Object}, \texttt{Array},
etc.).

Since this implementation approach relies on multiple runtimes of the
language, sending arbitrary objects between tasks can result in
unexpected behavior, e.g., when the object contains references.
%
As discussed in Section~\ref{sec:concrete}, we need to restrict |send|
to expressions that can be marshalled as strings, i.e., structurally
clonable objects.
%
In our formalization, this amounts to restricting the IFC language rule
for |send| such that only strings can be shared:
%{
\newcommand{\str}{"string"}
%format tOf (e) = "\texttt{typeOf}("e")\texttt{ === \str}"
%format ttrue = "\texttt{true}"
\begin{mathpar}
\inferrule[JS-send]
{
|il canFlowTo il'|\\
|iS(id') = Q|\\
|iS' = iS [ mapsto id' (il', id,  iv) , Q ]|\\
|ie = IT te|\\
|conf tS (tOf te) -> conf tS ttrue|
}
{|
iconf iS (fullconf id il tS (iniEi (send id' il' iv)), ldots)
.->
iS'; sched S (fullconf id il tS (iniEi unit), ldots)
|}
\end{mathpar}
%}
We remark that this is similar to the existing
\texttt{postMessage} API used for iframe and worker
communications~\cite{webworkers}.
%
Thus, the SWAPI implementation provides an IFC version of
\texttt{postMessage}, defined in terms of |send|, that additionally
takes a label argument.

Unsurprisingly, our coarse-grained combined language approach has been
inspired by existing browser (security) architectures.
%
Browsers, for example, isolate pages of different origins by running
them in separate runtimes.
%
Similarly, they provide the worker JS object~\cite{webworkers}, which allows
JavaScript to execute code in separate threads with separate JS
runtimes (and fresh global objects).
%
In both cases, code relies on the \texttt{postMessage} message-passing
API for communication, similar to our system.\footnote{
  The message-passing approach is, in part, due to ECMAScript's lack
  of well-defined semantics for concurrency.
  %
  Hence sharing and accessing objects such as the DOM across worker
  threads is undefined.
}
%
As suggested by our formal semantics, SWAPI can directly leverage these
mechanisms to implement the semantics of |specLangJS roundrobinf|
without modifying the JS runtime or intrusively changing the browser
layout engine.

However, a simple implementation of just associating a label with
workers and browsing context (iframes and top-level pages) to enforce
IFC on the executing JavaScript is not sufficient, as APIs which
introduce extra computational effects must be manually made aware of information flow
control.
%
For example, the initial global object of a worker contains the
\texttt{XMLHttpRequest} (XHR) object, while the global object of a
browsing context additionally contains the DOM.
%
In both cases, JS code can trivially leak information (e.g., to the
network with XHR or persistent storage using the DOM).
%
To address this, SWAPI additionally uses content security policy (CSP) and
(iframe) sandboxes~\cite{csp1.1,html5} to restrict external effects according
to the label of a task (worker or browsing context).  Accommodating these
features lies outside the purview of our formalism.

%% \Red{
%% EZY:  This seems like useless detail about the SWAPI implementation
%% that is not relevant to this paper.
%% 
%% Of course, simply disallowing all external effects for all browsing
%% contexts would break most of the Web.
%% %
%% Hence SWAPI only enforces IFC, and thus restricts external effects,
%% when the JS code uses the new API.\footnote{
%%   Once a piece of code ``opts-in'' to use our IFC API, in addition to
%%   restricting external effects we interpose ``standard''
%%   \texttt{postMessage} communication to ensure that no information is
%%   leaked between contexts that have opted-in and those that have not.
%% }
%% %
%% Moreover, rather than disallowing all external effects, SWAPI allows
%% controlled network communication by leveraging the fact that web
%% pages already have an accompanying security policy: the same origin
%% policy.
%% %
%% Hence, for example, a worker that has read data only sensitive to
%% \texttt{http://bank.ch} can communicate with this domain; only once
%% \texttt{http://aws.com} data is also included in the context will all
%% network requests be blocked.
%% %
%% Of course, this requires that we use a concrete label format to
%% express the policies; we refer the interested reader to~\tocite{} for
%% more details.}
