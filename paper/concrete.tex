\section{Concrete implementation}
\label{sec:concrete}

The operational semantics we have derived in the previous section
constitutes an abstract \emph{specification}: a semantics for which
non-interference is (obviously) true.  This specification can guide the
design of an information-flow calculus which would not be directly
derivable from our procedure, but is related to some such derivation.
For example, in our mini-ML language, the combined semantics requires
each thread to have a separate ML heap.  In an actual implementation, we
might like to only have a \emph{single} heap, with some restrictions to
enforce the illusion of separate heaps.  If these restrictions cause our
implementation semantics to behave in the same way as the specification
semantics (modulo some extra conditions), we might hope that we also
automatically achieve non-interference.

We can make this formal as follows: \Red{Define some notation}

\begin{theorem}
    Suppose we have a abstract specification language $L$ with machine
    states $|C|$, evaluation relation $|.->|$ and erasure function $e$.
    Furthermore we have a proof that this language fulfills termination
    sensitive non-interference: if $|C1| \approx_l |C2|$ and $|C1|
    |.->|^* |C1'|$, then there exists a $|C2'|$ such that $|C2| |.->|^*
    |C2'|$ and $|C1'| \approx_l |C2'|$.

    Next, suppose we have a language $L'$ with machine states $|Q|$,
    evaluation relation $|.->|'$ and erasure function $e'$.  Given
    functions $f^{-1} : |C -> Q|$ and $f : |Q -> C|$ relating machine
    states between $L$ and $L'$, obeying the condition $f^{-1} \circ f =
    id$.  Then to show termination sensitive non-interference for $L'$,
    it suffices to show that for every evaluation rule $|Q| |.->|' |Q'|$
    it is either the case that $f(|Q|) |.->| f(|Q'|)$ or that
    \ldots
\end{theorem}

\begin{proof}
Our proof proceeds along the same lines as a traditional TSNI proof.
\ldots (wait for the proof to be done)
\end{proof}

To provide a sense of how these conditions work, we give an example
use of these theorem:

\begin{theorem}
    The language described in Figure X satisfies termination sensitive non-interference.
\end{theorem}

\begin{code}
\end{code}


\begin{figure}

\begin{mathpar}
\inferrule[C-evalCtx]
{|
coconf iS tS (cconf id il ie)
.->
coconf iS' tS (cconf id il' ie')
|}
{|
coconf iS tS (cconf id il (iniE ie), ldots)
.->
iS' ; tS ; sched step (cconf id il' (iniE ie'), ldots)
|}
\end{mathpar}

\caption{ML with a single heap}
\label{fig:comb}
\end{figure}

This version
