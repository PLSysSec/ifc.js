\section{The monadic interpretation}
\label{sec:monad}

In the previous sections, we have been primarily concerned with the
combination of operational semantics in order to automatically augment a
language with information flow control.  In this section, we present
a more abstract view of the situation, phrased in the language of monads
and monad transformers.  The key idea is that the computational effects
of the target language should be thought of as a \emph{monad transformer}
on top of the information flow control base monad.  Readers who are not
familiar with monads should feel free to skip this section.

Monads were originally proposed as a method for organizing computational
effects~\cite{Moggi:1991:NCM:116981.116984}.  One of the key questions
of interest was how to modularly build more complex computational
effects out of simpler ones.  One particular approach, the
\emph{incremental
approach}~\cite{Benton00monadsand,Liang95monadtransformers}, utilizes
monad transformers, which add an extra computation feature to a
pre-existing monad.  In the past, this approach has been criticized for
a number of failings, including the fact that the order in meaning of a
monad can depend on the order in which the monad transformers are
applied. The canonical example of this phenomenon is the
composition of the state monad and the error monad: depending on the
ordering of the two monad transformers, state is preserved or thrown away
upon an error.

These problems can be viewed as defects in a setting where we are
simply interested in combining effects.  However, in the setting of information
flow control, we are primarily interested in augmenting the
computational effects of a target language with information flow control
\emph{without} compromising non-interference.  When the base monad is
kept abstract (in the style of a restricted IO monad~\cite{Terei:2012:SH:2364506.2364524}),
there is an easy argument why any monad transformer must also preserve non-interference:
the monad transformer could have been implemented simply as a \emph{pure} interpreter, and
the pure interpreter is just another program like any other that may run on top of
a restricted IO monad.

Now, one shouldn't simply force an interpreter to be reimplemented on
top of a language supporting information flow control.  As suggested by
Filinski et al~\cite{Filinski:2010:MA:1707801.1706354}, we should use
pure implementations of computational effects to understand
computational effects when built into information flow languages: they
give a good hint what the operational rules should be.  This intuition
backed the development of Section~\ref{sec:retrofit}.  Furthermore,
developing the relationship between concrete semantics and abstract
semantics in Section~\ref{sec:concrete} relates to the question of monad
isomorphisms in the presence of data abstraction.  We leave the
investigation of this connection in more detail to future work.
