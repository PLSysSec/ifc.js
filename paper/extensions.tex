\section{Extensions}
\label{sec:extensions}
\label{sec:extensions:labeled}

%In this section we consider several extensions to our IFC system.
%%
%Specifically, we discuss how the specification language can be
%extended to consider other IFC-aware features (e.g., fine-grained
%labeled values), and talk about a static type system, as well as
%more complex scheduling policies.
%%
%When considering these extensions we also discuss some limitations
%of our approach.

While the IFC language presented thus far provides the basic
information flow primitives, actual IFC implementations
may wish to extend the minimal system with more specialized
constructs.
%Different concrete IFC implementations may wish to extend our minimal
%system with labeled constructs appropriate for their envisioned
%applications.
%%
For example, COWL provides a labeled version of the XHR object, while
LIO implements a labeled file system, used in server-side web
applications~\cite{hails}.
%%
%Below we
%consider the addition of several labeled constructs to the abstract specification.
In this section we show how to extend the language with several such
constructs.  For space reasons, we mainly provide the high-level
ideas here, and provide full formal semantics (including the soundness
proof with the extensions) in \appendixextfirst{}.
%
%We remark that that the non-interference proofs must account for the
%new constructs, since they amend the IFC calculus semantics (usually
%by addition).

\subsection{Labeled values}
%{
%format Lb = "\textbf{Labeled}"
%format label = "\textbf{label}"
In traditional language-based dynamic IFC systems, a label is
associated with values.
%
Hence, a program that, for example, simply writes labeled messages to
a labeled log can operate on both public and sensitive values.
Similarly, a task that receives a sensitive value and forwards it
to another task does not have be be at a sensitive level, if the
value is not inspected.
%
In its simplest form, our coarse grained system requires that the
current label of a task be at least at the level of the sensitive data
to reflect the fact that such data is in scope.

If such fine-grained labeling of values is required, our base IFC
system can be extended with explicitly labeled
values, much like those of LIO and
Breeze~\cite{lio, Hritcu:2013:YIB:2497621.2498098}: |iv ::= cdots
inlinesep Labeled il ie|.
%
Following LIO, we say that the expression |ie| is protected by label |il|,
while the label |il| itself is protected by the task's current label.
%
%
The label of such values can be inspected the task without
requiring the current label to be raised.
%
However, when a task wishes to inspect the protected value |ie|, it
must first raise its label to at least |il| to reflect that it is
incorporating data at such sensitivity level in its scope.
%
When creating labeled values the label |il| must be above
the current label; otherwise it cannot be said that protection has
been transferred from the current label to |il|.
%
%}

\subsection{Labeled mutable references/variables/channels}
%
Extending the calculus with other labeled features, such as
references, mutable variables (MVars)~\cite{CH96}, or channels,
can be done in a similar manner: these references are implemented
in the IFC language, separately from any preexisting notions of
mutable references in the target language.
%
%While we can modify the IFC calculus to add such labeled state to each
%individual task, adding it to the system state |iS| leads to more
%interesting use cases.
%
There is some minor additional state to track: specifically, by amending |iS|, as in~\cite{lio,
stefan:addressing-covert}, we can allow threads to use these
constructs to synchronize, or communicate with constructs other than
|send|/|recv| in a safe manner.
%
For example, when extending the calculus with labeled references, |iS|
additionally contains a store that maps addresses to a value
and a label
which can be read and written to by different tasks through a labeled
reference implementations.

\subsection{Clearance}
%
Systems like LIO, COWL, and Breeze additionally provide a discretionary access
control (DAC) mechanism---called \emph{clearance}---at the language
level~\cite{Hritcu:2013:YIB:2497621.2498098, lio}.
%
This mechanisms is used to restrict a computation from allocating and
accessing data (or communicating with entities) above a specified
label, the clearance.
%
Amending our IFC language with clearance is straight forward,
and, can be done using our notation of a restricted language.
%
To this end, we first extend tasks to track a clearance label
alongside the current label, and amend the core IFC language with two
new terminals for retrieving and setting this value.
%
Since this extension only adds a per-task mutable variable whose value
has no influence on the system, all security guarantees still
hold, by essentially the same proofs.
%
However, this does not implement any DAC mechanism yet.
%
To do so, we can restrict the language with a family of predicates
$\mathcal{P}_\text{clearance}$:
All rules that
raise the current label (e.g., \textsc{I-setLabel}), perform
allocation (e.g., \textsc{I-sandbox} and \textsf{I-send}), or set the
clearance (clearance should not be arbitrarily raised), a predicate
that uses the clearance to impose DAC is used.
%
For instance, the predicate for \textsc{I-setLabel} prevents the
current label from being raised above the clearance (and thus permit
reads above the clearance).  The predicate $P := |il
canFlowTo il'|$ achieves this restriction, where |il'| is the
clearance and |il| is the current
label.
%
The other predicates are defined in a similar way and omitted for
brevity.


\subsection{Privileges}
Decentralized IFC extends IFC with the decentralized label model of
Myers and Liskov~\cite{myers:dlm} to allow for more general
applications, including systems consisting of mutually distrustful
parties.  In a decentralized system, a computation is executed with a
set of \emph{privileges}, which, when exercised, allow the computation
to declassify data (e.g., by lowering the current label).
%
Practical IFC systems
(e.g.,~\cite{Zeldovich:2006, lio,
  Hritcu:2013:YIB:2497621.2498098, myers:jif}) rely on privileges to
implement many applications.
%
%Since both the LIO and COWL implementations already support such features,
%we believe that extending our calculus to consider privileges is
%straight forward.
%
The challenge with such an extension lies in the precise
security guarantees that must be proved, which to the best of our
knowledge is an open research problem.

Our implementation for Node.js and COWL both provide privileges, but
we have not formalized this part any further.
%
%Addressing this problem is a natural direction for this work.


\section{Limitations and Future Directions}

In this section we state some limitations of the current system, and
outline potential future directions to alleviate these limitations.

\subsection{Type Safety}
\label{sec:extensions:types}

One important consideration for any new language feature is whether
or not it preserves type safety.  If implementations are allowed to
have undefined behavior when the system gets stuck, concerns of type
safety can be directly applicable to security.

In our presentation, we have demanded that implementations \emph{not}
have undefined behavior when getting stuck: in such situations, the
\textsc{I-noStep} rule applies, where the stuck thread should be dropped
from execution.  Our combined language will be not get stuck, even if
the target language could get stuck.  This sort of guarantee can be
achieved at a coarse granularity for languages like C by using hard
isolation.  However, for many other languages, this is too stringent a
requirement.

A possible way to relax this requirement is to drop the \textsc{I-noStep}
rule, and instead demand type-safety of the combined language.  This
depends on the type system of the target language, so it is difficult to
make any general statements about how one would go about doing this.  However,
there are few general remarks to be made:

\begin{itemize}
    \item In Matthews and Findler's original
        paper~\cite{Matthews:2007:OSM:1190216.1190220}, an emphasis was
        on using type boundaries to mediate between the type systems of
        the two languages, i.e. handling conversions of values from
        one language to the other.  These techniques are directly applicable here.

    \item In the case of mini-ES, type safety proceeds by a preservation theorem
        that refers to well-typed stores.  In our setting, there are now multiple
        stores, and expressions may move from being interpreted with one store
        to another (e.g. when an address is sent from one thread to
        another); a clear violation of type safety.  In this case, operations
        such as |send| have their typing rules require
        the expressions being sent be typeable in an arbitrary well-typed store,
        statically disallowing the sending of addresses, or any types which
        may contain addresses.  In practice, this may be further restricted to
        types which can be easily marshalled, e.g. strings.

    \item Similarly, the choice of $\kappa$ (from Section~\ref{sec:retrofit}) directly
        influences the difficulty of the type-safety proof for
        |sandbox| when the target language has a store.  When
        $\kappa$ is identity, most expressions can be allowed, since the store
        remains the same; if $\kappa$ drops the store, however, only closed expressions
        can be sandboxed.
\end{itemize}

\subsection{Scheduling Policies}
Our specification language is parametrized by a scheduling policy |alpha|, which
maps a task list to a potentially different task list, i.e.,
$\alpha_{{\tiny\mathrm{\text{step}}}}(\ifc{\Varid{t}},\ldots)$. 
%
As previously discussed, the precise definition of |alpha|
dictates whether the language is TSNI, TINI, or neither.
%
We described two popular policies |roundrobinf| and
|seqf| for which we showed TSNI and TINI, respectively.
%
However, our definitions allow for other scheduling policies based on the
current task list---in fact, our formalism can be easily 
extended to consider history~\cite{BartheRRS07}.
%
As an example of generality, our approach allows to implement the scheduler of~\cite{Kashyap:2011}
which always schedules less sensitive threads first; this can be
employed to address external timing covert channels where an attacker
can measure the delay between output events.\footnote{
  When considering a label lattice that is not totally ordered, e.g.,
  DLM~\cite{myers:dlm} or DCLabels~\cite{dclabels}, this scheduling policy is
  considerably more complex.
}
%
More precisely, our calculus generalizes to a class of deterministic
schedulers that are: 1) \emph{integrity-preserving}, i.e., they do not
modify (or create) tasks; 2) \emph{non-interferent}, i.e., 
%greatest lower bound of all the labels on the task labels,\footnote{
%  The label on a task's current label is the current label of the
%  parent at the time of creation, i.e., the initial current label of
%  the task.
%} 
scheduling decisions on public tasks do not depend on a sensitive task's
state~\cite{russo2006securing}; and, 3) make \emph{low-progress}, i.e., if there
is a task in the task list whose label is below (or incomparable to) the label
of another task, this task (or another one at the same level) must eventually
run.
%
The last condition is required for TSNI; languages relying on
schedulers, such as |seqf|, that do not satisfy the low-progress
condition can at most satisfy TINI.
%
For simplicity, we do not treat this, more-general class of schedulers;
doing so would require modifying our calculus to keep track of the
label on the current label of tasks and complicate the
non-interference proofs.

More fundamentally, our scheduler parametrization is limited in being
deterministic, and thus extending the system to consider non-deterministic
schedulers (or even non-deterministic target languages) is non-trivial.
%
%For instance,  we do not consider  schedulers that, among other, rely
%on the global state |iS| or previous task configurations, as treated
%in~\cite{russo2006securing}.
%
%Additionally, our definitions rely on deterministic relations,
%
Such extensions often apply security conditions which suffer 
from refinement attacks (e.g.~\cite{Smith:Volpano:MultiThreaded}) or may require to
adapt non-trivial definitions as
\emph{low-determinism}~\cite{Zdancewic:Myers:CSFW03,Huisman:Worah:Sunesen:CSFW06,Terauchi:2008},
which states that a program is secure only if |l|-equivalent results are
deterministic.

\subsection{External Effects}
\label{sec:extensions:external}
Our embedding assumes that the target language not have any
primitives that can induce external-world effects.
%
As discussed in Section~\ref{sec:real}, imposing this restriction
can be challenging.
%
Yet, external effects are crucial when implementing more complex
real-world applications.
%
For example, code in an IFC browser must load resources or
perform XHR to be useful.

Like labeled references, features with external effects must be
modeled in the IFC language; we must reason about the precise security
implications of features that otherwise inherently leak data.
%
Previous approaches have modeled external effects by internalizing the
effects as writes/reads to labeled channels/references~\cite{stefan:addressing-covert}.
%
An alternative approach is to model such effects as messages to/from
certain labeled tasks.
%
These ``special'' tasks are trusted with access to the unlabeled
primitives that can be used to perform the external effects; since the
interface to these tasks is already part of the IFC language, the
proof only requires showing that this task does not leak information.
%
This latter approach of wrapping unsafe primitives is, for example,
used in COWL to allow for controlled network communication.
%
By wrapping the default XHR object, for example, we can allow code
to communicate with hosts according to the task's current label.
